\section{Динамическое программирование}

\textit{Динамическое программирование} --- способ решения сложных задач путём разбиения на простые подзадачи.

Чтобы успешно решить задачу методом динамического программирования в общем случае
нужно продумать пять пунктов:
\begin{enumerate}
\item Что храниться в качестве значения динамики и какие параметры однозначно определяют её состояние;
\item Начальные значения динамики;
\item Зависимости между состояниями --- формула пересчёта значений динамики;
\item Порядок пересчёта;
\item Как вычисляется итоговый ответ --- это могут быть какие-то значения посчитанной динамики или функция, использующая их.
\end{enumerate}

Существует три основных порядка пересчёта:
\begin{enumerate}
\item Прямой порядок --- состояние динамики пересчитывается из уже посчитанных;
\item Обратный порядок --- из текущего состояния динамики обновляются зависящие от него;
\item <<Ленивая динамика>> --- состояние пересчитывается рекурсивно с запоминанием значений динамики.
\end{enumerate}